%%%%%%%%%%%%%%%%%%%%%%%%%%%%%%%%%%%%%%%%%%%%%%%%%%%%%%%%%%%%%%%%%%%%%%%%%%%%%%%%
% AMS Beamer series / Bologna FC / Template
% Andrea Omicini
% Alma Mater Studiorum - Università di Bologna
% mailto:andrea.omicini@unibo.it
%%%%%%%%%%%%%%%%%%%%%%%%%%%%%%%%%%%%%%%%%%%%%%%%%%%%%%%%%%%%%%%%%%%%%%%%%%%%%%%%
%\documentclass[handout]{beamer}\mode<handout>{\usetheme{default}}
%
\documentclass[presentation]{beamer}\mode<presentation>{\usetheme{AMSBolognaFC}}
%\documentclass[handout]{beamer}\mode<handout>{\usetheme{AMSBolognaFC}}
%%%%%%%%%%%%%%%%%%%%%%%%%%%%%%%%%%%%%%%%%%%%%%%%%%%%%%%%%%%%%%%%%%%%%%%%%%%%%%%%
\usepackage[T1]{fontenc}
\usepackage{wasysym}
\usepackage{amsmath,blkarray}
\usepackage{centernot}
\usepackage{fontawesome}
\usepackage{fancyvrb}
\usepackage[ddmmyyyy]{datetime}
\renewcommand{\dateseparator}{}
%\renewcommand{\thefootnote}{\fnsymbol{footnote}}
\newcommand{\version}{1}
\usepackage[
	backend=biber,
	%citestyle=authoryear-icomp,
	maxcitenames=1,
	bibstyle=alphabetic]{biblatex}

	\makeatletter

\addbibresource{bibliography.bib}
%%%%%%%%%%%%%%%%%%%%%%%%%%%%%%%%%%%%%%%%%%%%%%%%%%%%%%%%%%%%%%%%%%%%%%%%%%%%%%%%
\title[]
{Hybrid AI-based engineering of cyber-physical swarms}
%
\subtitle[Research Project Proposal]
{Research Project Proposal}
%
\author[\sspeaker{Domini}]
{\speaker{Davide Domini} \href{mailto:davide.domini@studio.unibo.it}{davide.domini@studio.unibo.it}}
%
\institute[DISI, Univ.\ Bologna]
{Department of Computer Science and Engineering - DISI\\\textsc{Alma Mater Studiorum} -- University of Bologna
\\[0.5cm]
\textbf{Ph.D. Programme in Computer Science And Engineering \\ Admission XXXIX Cycle}}

%
\renewcommand{\dateseparator}{/}
\date[\today]{\today}
%
%%%%%%%%%%%%%%%%%%%%%%%%%%%%%%%%%%%%%%%%%%%%%%%%%%%%%%%%%%%%%%%%%%%%%%%%%%%%%%%%
\begin{document}
%%%%%%%%%%%%%%%%%%%%%%%%%%%%%%%%%%%%%%%%%%%%%%%%%%%%%%%%%%%%%%%%%%%%%%%%%%%%%%%%

%/////////
\frame{\titlepage}
%/////////

%%===============================================================================
\section*{Outline}
%%===============================================================================

%%/////////
\frame[c]{\tableofcontents[hideallsubsections]}
%%/////////

%===============================================================================
\section{Introduction}
%===============================================================================

%/////////
\begin{frame}[allowframebreaks]{Introduction}
%/////////

\begin{block}{Cyber-Physical Swarms (CPSW)}
	\begin{itemize}
		\item A myriad devices that interact with the environment and exchange information 
			among themselves;
		%\item Focus on properties like self-organization %\cite{schmeck2011organic}
		%	 and collective intelligence;%\cite{tumer2004survey};
		\item A wide range of applied domains, including: smart cities \cite{zedadra2019swarm}, 
			swarm robotics \cite{brambilla2013swarm}, 
			large-scale IoT systems \cite{uslu2023role}, and more.
	\end{itemize}
\end{block}

\begin{alertblock}{Engineering approaches}
	\begin{itemize}
		\item Macro-programming: manually developing controllers 
			from a global perspective (e.g., \emph{Aggregate Computing} \cite {viroli2018field});
		\item AI: learn directly from experience and/or data 
			(e.g., \emph{Multi-Agent Reinforcement Learning} \cite{4445757} and 
			\emph{Graph Neural Networks} \cite{9046288}).
		\item Open space for \emph{hybrid approaches}.
	\end{itemize}
\end{alertblock}

\end{frame}
%/////////


%===============================================================================
\section{State of the art}
%===============================================================================

%/////////
\begin{frame}[allowframebreaks]{Aggregate computing}
%/////////
\begin{columns}
\begin{column}{0.55\textwidth}
	\begin{block}{Overview}
		\begin{itemize}
			\item \emph{Computational fields} \cite{mamei2004cofields, viroli2019distributed} as first-class abstractions
			\item Programs as field transformations through \emph{field calculus} \cite{viroli2016higher}
		\end{itemize}
	\end{block}
\end{column}
\begin{column}{0.45\textwidth}
\includegraphics[width=\textwidth]{img/ac.png}
\end{column}
\end{columns}

\centering
\begin{alertblock}{Computational model}
\begin{itemize}
	\item Each node shares the same \emph{aggregate program}
	\item Each node execute rounds iteratively and asynchronously:
	\begin{enumerate}
		\item \textbf{Context building}: collect information from the neighborhood and sensors
		\item \textbf{Program execution}: execute the aggregate program on the context
		\item \textbf{Export sharing}: share the export with the neighborhood
	\end{enumerate}
\end{itemize}
\end{alertblock}

\begin{alertblock}{Advantages}
	\begin{itemize}
		\item Programs defined in a \emph{composable} and \emph{declarative} manner;
		\item Promotes the reuse of behaviours;
		\item Low-level aspects (e.g., failures) are automatically handled by the middleware;
		\item A reliable and efficient Scala implementation: \emph{ScaFi} \cite{casadei2022scafi}
		\begin{itemize}
			\item A \emph{domain specific language (DSL)} for specifying aggregate computation;
			\item A \emph{simulation environment}, through the Alchemist simulator \cite{alchemist};
			\item A \emph{middleware} for executing and deploying aggregate programs;
			\item Reusable \emph{library functionalities} that serve as building blocks for constructing new aggregate
			programs (e.g., Gradients).
		\end{itemize}
	\end{itemize}
\end{alertblock}
	
\end{frame}
%/////////

%/////////
\begin{frame}[c]{Learning from experience}
%/////////
	
\end{frame}
%/////////

%/////////
\begin{frame}[c]{Graph Neural Networks}
%/////////
	
\end{frame}
%/////////



%===============================================================================
\section{Project description}
%===============================================================================

%/////////
\begin{frame}[c]{Activities}

	
\end{frame}
%/////////

%/////////
\begin{frame}[c]{Scope}

\end{frame}
%/////////


%/////////
\begin{frame}[c]{Expected results}

\end{frame}
%/////////



%===============================================================================
\section*{}
%===============================================================================

%/////////
\frame{\titlepage}
%/////////

%===============================================================================
\section*{\refname}
%===============================================================================

%%%%
\setbeamertemplate{page number in head/foot}{}
%/////////
\begin{frame}[c,noframenumbering, allowframebreaks]{\refname}
%\begin{frame}[t,allowframebreaks,noframenumbering]{\refname}
	\tiny
	\nocite{*}
	\printbibliography
\end{frame}
%/////////

%%%%%%%%%%%%%%%%%%%%%%%%%%%%%%%%%%%%%%%%%%%%%%%%%%%%%%%%%%%%%%%%%%%%%%%%%%%%%%%%
\end{document}
%%%%%%%%%%%%%%%%%%%%%%%%%%%%%%%%%%%%%%%%%%%%%%%%%%%%%%%%%%%%%%%%%%%%%%%%%%%%%%%%
